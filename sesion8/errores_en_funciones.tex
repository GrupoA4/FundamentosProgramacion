\documentclass[12pt]{article}

\usepackage{lmodern}
\usepackage[T1]{fontenc}
\usepackage[spanish,activeacute]{babel}
\usepackage{listings}

\title{Errores en funciones}
\date{}
\setlength{\parskip}{2em}
\pagestyle{plain}

\begin{document}
\maketitle

\section*{a)}

\lstset{language=C++, breaklines=true, basicstyle=\footnotesize}
\begin{lstlisting}
int ValorAbsoluto (int entero) {
	if (entero < 0)
		entero = -entero;
	else
		return entero;
}
\end{lstlisting}

No es correcto retornar la variable de entrada.

\begin{lstlisting}
int ValorAbsoluto (int entero) {
	int valor_absoluto = entero;
	if (entero < 0)
		valor_absoluto = -entero;
	else
		return valor_absoluto;
}
\end{lstlisting}

\section*{b)}

\begin{lstlisting}
void Cuadrado (int entero){
	return entero*entero;
}
\end{lstlisting}

No es correcto devolver directamente el resultado de
una operaci'on.

\begin{lstlisting}
void Cuadrado (int entero){
	int cuadrado;
	cuadrado = entero*entero;
	return cuadrado;
}
\end{lstlisting}

\section*{c)}

\begin{lstlisting}
void Imprime (double valor){
	double valor;
	
	cout << valor;
}
\end{lstlisting}

No tiene sentido declarar otra variable con el mismo nombre.

\begin{lstlisting}
void Imprime (double valor){
	cout << valor;
}
\end{lstlisting}

\section*{d)}

\begin{lstlisting}
bool EsPositivo(int valor){
	if ( valor > 0)
		return true;
}
\end{lstlisting}

No es bueno utilizar la sentencia return dentro de estructuras.

\begin{lstlisting}
bool EsPositivo(int valor){
	bool es_positivo = false;
	
	if ( valor > 0)
		es_positivo = true;
		
	return true;
}
\end{lstlisting}
	
\end{document}